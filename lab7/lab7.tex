\documentclass[a4paper,14pt]{extarticle}

\usepackage[top=1in, bottom=1in, left=1in, right=1in]{geometry}
\usepackage[utf8]{inputenc}
\usepackage[russian]{babel}
\usepackage{graphicx}
\usepackage{caption}
\usepackage{subcaption}
\usepackage{chngcntr}
\usepackage{amsmath}
\usepackage{amsfonts}
\usepackage{pgfplots}
\usepackage{pgfplotstable}
\usepgfplotslibrary{fillbetween}
\usepackage{float}
\usepackage{lipsum}% http://ctan.org/pkg/lipsum
\usepackage{multicol}% http://ctan.org/pkg/multicol
\usepackage{hhline}
\usepackage{tabularx}
\usepackage{tikz,xcolor}
\usepackage{tkz-graph}
\usepackage{float}
\usepackage{mathtools}
\usepackage{todonotes}
\usepackage{listings}
\usepackage[makeroom]{cancel}

\usetikzlibrary{arrows, petri, topaths}

\counterwithin{figure}{section}
\counterwithin{equation}{section}
\counterwithin{table}{section}

\begin{document}
\begin{titlepage}
\centering 
{\bfseries Санкт-Петербургский Политехнический Университет} \\
Институт компьютерных наук и технологий \\
Кафедра компьютерных систем и программных технологий \\
\vspace{5cm}
{\centering \textbf{Отчёт о лабораторной работе 7} \\ 
\vspace{0.2cm}
\textbf{Дисциплина}: Телекоммуникационные технологии \\
\vspace{0.2cm}
\textbf{Тема}: Помехоустойчивые коды. } \\
\vspace{4cm}
\hfill {\bfseries Работу выполнил:}  \\
\hfill гр. 33501/3 Кнорре А.В. \\
\hfill {\bfseries Преподаватель}  \\
\hfill Богач Н.В.
\vfill
Санкт-Петербург \\
{\large 2018}
\end{titlepage}

\section{Цель работы}
Изучение методов помехоустойчивого кодирования и сравнение их свойств:
\begin{itemize}
\item Провести кодирование/декодирование сигнала, полученного с помощью функции randerr кодом Хэмминга 2-мя способами: с помощью встроенных функций encode/decode, а также через создание проверочной и генераторной матриц и вычисление синдрома. Оценить корректирующую способность кода. 
\item Выполнить кодирование/декодирование циклическим кодом, кодом БЧХ, кодом Рида- оломона. Оценить корректирующую способность кода.
\end{itemize}

\section{Ход работы}

\subsection{Код Хэмминга}
- это алгоритм, который позволяет закодировать какое-либо информационное сообщение определённым образом и после передачи (например по сети) определить появилась ли ошибка в этом сообщении (к примеру из-за помех) и, при возможности, восстановить это сообщение. Он является подклассом циклических кодов, в которых перестановка символов в кодированном блоке дает другой кодированный блок того же кода (не изменяет результирующий после обработки код).

\subsection{Matlab}

Применим Код Хэмминга чтобы протестировать детектирование ошибок в сообщении с помехами:
\begin{verbatim}
% gen2par(G);
msg = [1 0 0 1]
code=encode(msg, 7,4) % 0 1 1 1 0 0 1
code = [1 1 1 1 0 0 1] % added 1 error
decode(code,7,4) % 1 0 0 1 (error corrected)

code = [1 0 1 1 0 0 1] % added 2 errors
decode(code,7,4) % 0 0 0 1 (error occurred in bit 0)
\end{verbatim}
Видим что код справился лишь с одной ошибкой, что согласуется с теорией.

Создадим синдром вручную:
\begin{verbatim}
msg = [1 0 0 1 1]
% create syndrome 
H = [
 1 0 0 1; %a4
 0 1 1 1; %a3
 0 1 1 0; %a2
 0 1 0 1; %a1
 0 0 1 1; %a0
 1 0 0 0; %b3 %E
 0 1 0 0; %b2
 0 0 1 0; %b1
 0 0 0 1]' %b0
a0 = msg(5)
a1 = msg(4)
a2 = msg(3)
a3 = msg(2)
a4 = msg(1)
b0 = mod(a4, 2)
b1 = mod(a1 + a2 + a3, 2)
b2 = mod(a0 + a2 + a3, 2)
b3 = mod(a0 + a1 + a3 + a4, 2)
code = [msg, b3, b2, b1, b0]
mod(code * (H'), 2)% 0 - no error

code = [1 0 0 1 1 1 1 1 0]
mod(code * (H'), 2) % 1 - error in bit 1
\end{verbatim}

Наблюдаем как обнаруживается ошибка с помощью умножения на транспонированную матрицу синдрома 

\section{Выводы.}

В данной работе мы познакомились с Кодом Хэмминга и его способностью по обнаружению ошибок в посылках.


\end{document}
